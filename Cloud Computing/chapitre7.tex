\chapter{Neutron Configuration}
%Intro\footnotemark\\
\begin{spacing}{1.2}
%note en bas de page
\section{Neutron Setup in Keystone}
\subsection{Adding user or service for Neutron on Keystone}
\par We will start by creating a new Neutron user, assigning him the role of admin,

\\
\begin{figure}[!htb] 
\begin{center} 
\includegraphics[width=1\linewidth]{Cloud/Neutron Setup in Keystone/create [neutron] user in [service] project} 
\end{center} 
\caption{create [neutron] user in [service] project} 
\end{figure} 
\FloatBarrier
\\
\begin{figure}[!htb] 
\begin{center} 
\includegraphics[width=1\linewidth]{Cloud/Neutron Setup in Keystone/add [neutron] user in [admin] role}
\end{center} 
\caption{add [neutron] user in [admin] role} 
\end{figure} 
\FloatBarrier

\par creating a service entry for it, define Neutron API as a host.
\\
\begin{figure}[!htb] 
\begin{center} 
\includegraphics[width=1\linewidth]{Cloud/Neutron Setup in Keystone/define Neutron API Host} 
\end{center} 
\caption{define Neutron API Host} 
\end{figure} 
\FloatBarrier

\par creating an endpoint for the interfaces
public, internal and admin: 
\\
\begin{figure}[!htb] 
\begin{center} 
\includegraphics[width=1\linewidth]{Cloud/Neutron Setup in Keystone/create endpoint for [neutron] (admin)} 
\end{center} 
\caption{create endpoint for [neutron] (admin)} 
\end{figure} 
\FloatBarrier

\subsection{Adding a User and Database on MariaDB for Neutron}
\par Next, we'll add this new user to our mariadb database: 
\\
\begin{figure}[!htb] 
\begin{center} 
\includegraphics[width=1\linewidth]{Cloud/Neutron Setup in Keystone/Add a User and Database on MariaDB for Neutron} 
\end{center} 
\caption{Add a User and Database on MariaDB for Neutron} 
\end{figure} 
\FloatBarrier


\section{Installing and Configuring Neutron services}
\subsection{Installing Neutron services}

\par Now we will install the services of Neutron to configure them later 
\\
\begin{figure}[!htb] 
\begin{center} 
\includegraphics[width=1\linewidth]{Cloud/Installing and Configuring Neutron services/Install Neutron Services 1} 
\end{center} 
\caption{Install Neutron Services 1} 
\end{figure} 
\FloatBarrier
\\

\subsection{Configuring Neutron services}
\par For the configuration, we will rename the file /etc/neutron/neutron.conf.org in \newline
/Etc/neutron/neutron.conf. Here is the contents of the file: 
\\
\begin{figure}[!htb] 
\begin{center} 
\includegraphics[width=1\linewidth]{Cloud/Installing and Configuring Neutron services/Creating neutron.conf} 
\end{center} 
\caption{Creating neutron.conf} 
\end{figure} 
\FloatBarrier
\\
\par We will change the access permissions to this file with the chmod 640 command. Then we
let's define neutron as a group user.

\par In the / etc / neutron / l3 agent.ini file, we will add the following lines: 
\\
\begin{figure}[!htb] 
\begin{center} 
\includegraphics[width=1\linewidth]{Cloud/Installing and Configuring Neutron services/Changing l3_agent.ini} 
\end{center} 
\caption{Changing l3_agent.ini} 
\end{figure} 
\FloatBarrier
\\
\par In the / etc / neutron / dhcp agent.ini file, we will add the following lines: 
\\
\begin{figure}[!htb] 
\begin{center} 
\includegraphics[width=1\linewidth]{Cloud/Installing and Configuring Neutron services/Changing dhcp_agent.ini} 
\end{center} 
\caption{Changing dhcp_agent.ini} 
\end{figure} 
\FloatBarrier
\\
\par In the / etc / neutron / metadata agent.ini file, we will add the following code: 
\\
\begin{figure}[!htb] 
\begin{center} 
\includegraphics[width=1\linewidth]{Cloud/Installing and Configuring Neutron services/Changing metadata_agent.ini} 
\end{center} 
\caption{Changing metadata_agent.ini} 
\end{figure} 
\FloatBarrier
\\
\par In the file / etc / neutron / plugins / ml2 / ml2 conf.ini, we will add the following lines:
\\
\begin{figure}[!htb] 
\begin{center} 
\includegraphics[width=1\linewidth]{Cloud/Installing and Configuring Neutron services/Chaning ml2_conf.ini} 
\end{center} 
\caption{Chaning ml2_conf.ini} 
\end{figure} 
\FloatBarrier
\\
\par In the file / etc / neutron / plugins / ml2 / openvswitch agent.ini, we will add the code
following :
\\
\begin{figure}[!htb] 
\begin{center} 
\includegraphics[width=1\linewidth]{Cloud/Installing and Configuring Neutron services/Chaning openvswitch_agent.ini} 
\end{center} 
\caption{Chaning openvswitch_agent.ini} 
\end{figure} 
\FloatBarrier
\\
\par Finally in the /etc/nova/nova.conf file, we will add the following lines:
\\
\begin{figure}[!htb] 
\begin{center} 
\includegraphics[width=1\linewidth]{Cloud/Installing and Configuring Neutron services/Changing nova.conf} 
\end{center} 
\caption{Changing nova.conf} 
\end{figure} 
\FloatBarrier
\\

\subsection{Starting Neutron services}
\par Enabling the openvswitch service
\\
\begin{figure}[!htb] 
\begin{center} 
\includegraphics[width=1\linewidth]{Cloud/Installing and Configuring Neutron services/Enabling openvswitch service} 
\end{center} 
\caption{Enabling openvswitch service} 
\end{figure} 
\FloatBarrier
\\

\par Now, we will finally be able to launch the Neutron service: 
\\
\begin{figure}[!htb] 
\begin{center} 
\includegraphics[width=1\linewidth]{Cloud/Installing and Configuring Neutron services/Showing network agents} 
\end{center} 
\caption{Showing network agents} 
\end{figure} 
\FloatBarrier
\\

\section{Configuring Neutron Networking}
\subsection{Configuring Neutron services}
\par It's time to set up the network for Neutron. For this, we will chain these commands 
\\
\begin{figure}[!htb] 
\begin{center} 
\includegraphics[width=1\linewidth]{Cloud/Configuring Neutron Networking/add bridge} 
\end{center} 
\caption{add bridge} 
\end{figure} 
\FloatBarrier
\\
\par we have to add the following at the end of the ml2_conf.ini and openvswitch_agent.ini files.
\\
\begin{figure}[!htb] 
\begin{center} 
\includegraphics[width=1\linewidth]{Cloud/Configuring Neutron Networking/add to the end of ml2_conf} 
\end{center} 
\caption{add to the end of ml2_conf} 
\end{figure} 
\FloatBarrier
\begin{figure}[!htb] 
\begin{center} 
\includegraphics[width=1\linewidth]{Cloud/Configuring Neutron Networking/add to the end of openvswitch-agent} 
\end{center} 
\caption{add to the end of openvswitch-agent} 
\end{figure} 
\FloatBarrier

\subsection{Creating virtual network}
\par We will then create a virtual network named sharednet1: 
\\
\begin{figure}[!htb] 
\begin{center} 
\includegraphics[width=1\linewidth]{Cloud/Configuring Neutron Networking/create network named [sharednet1]} 
\end{center} 
\caption{create network named [sharednet1]} 
\end{figure} 
\FloatBarrier

\par We are going to create a 10.0.0.0/24 subnet for the sharednet1 network: 
\\
\begin{figure}[!htb] 
\begin{center} 
\includegraphics[width=1\linewidth]{Cloud/Configuring Neutron Networking/create subnet [10.0.0.024] in [sharednet1]} 
\end{center} 
\caption{create subnet [10.0.0.024] in [sharednet1]} 
\end{figure} 
\FloatBarrier

\par Finally, we will confirm these parameters:
\\
\begin{figure}[!htb] 
\begin{center} 
\includegraphics[width=1\linewidth]{Cloud/Configuring Neutron Networking/confirm network list} 
\end{center} 
\caption{confirm network list} 
\end{figure} 
\FloatBarrier


\end{spacing}