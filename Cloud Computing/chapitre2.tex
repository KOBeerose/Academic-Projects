\chapter{Pre-Requirementsuration Setup}
\par Now whit our envirement ready to use, we will be installing Openstack Victoria that will help us to build our private cloud, RabbitMQ as message streaming, broker and messaging queue implementation that Openstack services will use in order to communicate since we are in the context of a distributed system. And finally Memcached to ensure that our system will have a caching protocol that fit our distributed system; caching is used to keep important and most demanded information fast to access an store it in memory rather than the hard-drive.
\begin{spacing}{1.2}
%note en bas de page
\section{Openstack Victoria}

\par The installation of centos-release-openstack-victoria, rabbitmq-server and memcached is done via dnf. A
After installation is complete, an update of the CentOS System is required. 
\\
\begin{figure}[!htb] 
\begin{center} 
\includegraphics[width=1\linewidth]{Cloud/Pre-Requirements/Installing openstack-victoria} 
\end{center} 
\caption{Installing openstack-victoria} 
\end{figure}  \FloatBarrier
\\
\\
\begin{figure}[!htb] 
\begin{center} 
\includegraphics[width=1\linewidth]{Cloud/Pre-Requirements/Add Openstack Repo _ Upgrade CentOS System} 
\end{center} 
\caption{Add Openstack Repo _ Upgrade CentOS System} 
\end{figure}  \FloatBarrier
\\

\section{Installation of RabbitMQ, Memcached.}

\par First of all we need to install RabbitMQ server
\\
\begin{figure}[!htb] 
\begin{center} 
\includegraphics[width=1\linewidth]{Cloud/Pre-Requirements/Installing rabbitmq-server} 
\end{center} 
\caption{Installing rabbitmq-server} 
\end{figure}  \FloatBarrier
\\
\par We are going to change in the file /etc/my.cnf.d/mariadb-server.cnf the default value 151
which is not sufficient in the Openstack environment. And to consider the modifications, we
let's restart and enable mariadb rabbitmq-server and memcached. 
\\
\begin{figure}[!htb] 
\begin{center} 
\includegraphics[width=1\linewidth]{Cloud/Pre-Requirements/changing max_connections} 
\end{center} 
\caption{changing max connections} 
\end{figure}  \FloatBarrier
\\
\par now we persue with configuring the memcached file in order to listen to all
\\
\begin{figure}[!htb] 
\begin{center} 
\includegraphics[width=1\linewidth]{Cloud/Pre-Requirements/memcached config} 
\end{center} 
\caption{memcached config} 
\end{figure}  \FloatBarrier
\\
\section{rabbitmqctl config }
\\
\par After that, we will add a new openstack user, define a password for him.
, also give it all the permissions and if SELinux is enabled, we have to change the policy
via a rabbitmqctl.te file and We will use the checkmodule and semodule commands to verify and compile this module.
of SELinux security policy in a binary representation8 
\\
\begin{figure}[!htb] 
\begin{center} 
\includegraphics[width=1\linewidth]{Cloud/Pre-Requirements/Creating a new openstack user} 
\end{center} 
\caption{Creating a new openstack user} 
\end{figure}  \FloatBarrier
\\


\par Allowing mysql service, the port 5672 service and reloading the firewall.\\
\\
\begin{figure}[!htb] 
\begin{center} 
\includegraphics[width=.8\linewidth]{Cloud/Pre-Requirements/Allowing ports for services.} 
\end{center} 
\caption{Allowing ports for services.} 
\end{figure}  \FloatBarrier
\\


\end{spacing}